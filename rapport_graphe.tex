\documentclass[draft]{scrartcl}
\usepackage[utf8]{inputenc}
\usepackage[frenchb]{babel}
\usepackage{lmodern}
\usepackage[T1]{fontenc}

\begin{document}
\title{Rapport du projet 1 de théorie des graphes}
\author{Maxence Ahlouche \and Maxime Arthaud \and Korantin Auguste
          \and Martin Carton \and Thomas Forgione \and Thomas Wagner}
\date{todo date}
\maketitle

\section{TPs} % TODO: meilleurs titre
    Nous avons commencé les TPs par le choix du langage utilisé pour
    implémenter nos algorithmes: le python, (pour la simplicité qu'il offre pour
    représenter les graphes et la lecture/le parsage d'un fichier représentant
    un graphe), puis l'écriture des classes permettant de représenter un
    graphe.

    Puis nous nous sommes séparés en trois binômes ayant travaillé sur:
    \begin{itemize}
      \item un algorithme permettant de savoir si un graphe est connexe ou non
        (dans le cas non orienté);
      \item  la création d'un algorithme (brute force) qui trouve un chemin
        hamiltonien dans un graphe;
      \item  la création de quelques fonctions qui génèrent des graphes simples
        de tests.
    \end{itemize}
    
\section{Problème 1}
  \subsection{Exercice 1}
    Pour représenter ce problème par un graphe, on représente les carrefours
    par les nœuds du graphe et les routes par des arêtes. Goudronner toutes les
    routes revient alors à parcourir toutes les arêtes une et une seule fois
    dans le graphe.

    Afin que le problème soit soluble, il faut que le graphe soit eulérien,
    c'est à dire connexe et que chaque nœud ait un nombre pair d'arêtes. Cette
    condition n'est pas nécessaire, car il peut toutefois aussi contenir
    exactement deux nœuds ayant un nombre impair d'arêtes, si on part d'un de
    ces deux nœuds (et on arrivera à l'autre).

    On peut utiliser l'algorithme d'Euler, qui consiste à trouver un cycle dans
    le graphe (en le parcourant «~au hasard~»), puis à s'appeler récursivement
    sur le sous-graphe construit en retirant les arêtes du cycle trouvé, en
    partant d'un nœud du cycle qui n'a pas d'arêtes qui n'appartiennent pas à
    ce cycle.

  \subsection{Exercice 2}
    On commence par construire le graphe représentant le musée où les arêtes
    représentent les portes et les sommets des salles du musée. Par définition,
    l'objectif de cet enfant est réalisable si ce graphe est hamiltonien ou
    semi-hamiltonien.

    Un graphe hamiltonien est un graphe qui contient au moins un cycle
    hamiltonien, c'est à dire un cycle passant une et une seule fois par chaque
    sommet en revenant au sommet de départ. S'il est uniquement
    semi-hamiltonien, il n'a qu'une chaîne hamiltonienne: l'enfant ne se
    retrouvera pas dans la salle de départ.

    Un moyen simple de résoudre ce problème est de tester toutes les
    possibilités de chaînes, jusqu'à en trouver une qui soit hamiltonienne.
    Toutefois, cette solution a une complexité en $O(n!)$ (avec $n$ le nombre
    d'arêtes), et par conséquent est peu envisageable pour des graphes de
    grande taille.

    Nous avons implémenté cet algorithme dans la fonction
    \verb+hamiltonian_path+.
    
    Pour améliorer cet algorithme, on peut arrêter de chercher
    quand le graphe restant à parcourir n'est pas connexe.

    Un autre moyen serait de calculer les puissances successives de la matrice
    latine représentant le graphe. Il suffirait alors d'éliminer tous les
    chemins ne contenant pas une et une seule fois tous les sommets du graphe.
    Lors du calcul des puissances successives, on peut remplacer par des 0 tous
    les chemins contenant deux fois le même sommet. Toutefois, cet algorithme
    est également très complexe.

    Ce problème étant NP-complet, il n'existe pas de moyen simple de déterminer
    le chemin, ni même si un tel chemin existe pour un graphe donné quelconque.

  \subsection{Exercice 3}
    Il est possible de résoudre le premier problème en transformant le graphe:
    il suffit de créer le graphe dont les sommets sont les arêtes du premier
    graphe, et les arêtes sont les sommets de ce premier graphe. Ce nouveau
    graphe est appelé «~line graph~». En trouvant une chaîne hamiltonienne
    dans ce graphe, on obtient une chaîne eulérienne.

    Par contre, il est impossible de transformer le problème 2 en problème 1.
    Le problème 2 est NP-complet, alors que le premier est soluble en $O(n)$ (à
    l'aide par exemple de l'algorithme de Hierholzer). Les 2 problèmes ne sont
    donc pas équivalents.

    %todo: parler des graphes orientés

\section{Problème 2}

%\section{Problème 3}
%  On en parle un peu juste pour faire bien ?

\end{document}

