%Rappel: L'objectif de ce rapport final (CRF6) est de
%         ***corriger les erreurs***
%commises dans les rapports (CRFs) précédents (ndlr: il y en a!) et de
%faire la preuve que les travaux de l'équipe l'ont conduite à
%         ***acquérir des connaissances théoriques***
% dans les divers domaines abordés.

Dans ce rapport, nous allons présenter les différents algorithmes, modèles,
théorèmes, etc.\  que nous avons pu découvrir ou approfondir lors de l'UE de
graphes et recherche opérationnelle. Cette UE était très riche et nous y avons
beaucoup appris. Mais ce rapport ne présentera pas exhaustivement ce que nous
avons fait.

En effet, il ne servira pas à présenter les travaux accomplis lors des
différentes unités d'acquisitions, ni à présenter nos programmes et les
résultats de nos tests (pour cela, se référer aux rapports respectifs des
différentes UAs), mais à présenter les connaissances théoriques que nous avons
acquises.

Il se terminera par un bilan de ce que nous avons acquis lors des différentes
UAs, accompagné de remarques sur notre travail en groupe.
