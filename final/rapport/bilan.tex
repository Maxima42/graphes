% Un bilan devra être présenté dans ce dernier rapport final sur la façon de
% travailler de l'équipe au cours de toutes les UAs. Une autoévaluation devra y
% être conduite de sorte à faire apparaitre les points positifs et/ou les
% disfonctionnements apparus dans l'équipe.

\subsection{Enseignement}

% wtf les séances de TD
Nous avons étés surpris de la façon dont fonctionnent
les séances de TD, et avons eu du mal à nous y faire:
lors des séances, nous étions censés être lâchés sur les
différents problèmes et nous renseigner par nous-mêmes.
Toutefois, le professeur présent se mettait souvent à présenter
des choses au tableau, et nous ne savions plus trop si nous
devions avancer, ou stopper tout travail et l'écouter.

% chacun a quasiment rien vu
Il a aussi été assez frustrant de se répartir le travail:
comme nous étions dans un groupe de 6 et avions quand même
pas mal de choses à faire, il fallait forcément se concentrer
sur un seul problème. Du coup, chacun a simplement abordé
quelques points précis, et n'a pas vraiment acquis de connaissances
poussées sur tout le reste des sujets abordés\dots

% rapports, rapports...
Le fait de devoir faire des rapports à chaque séance était aussi
très lourd à gérer, et assez peu pratique: nous avions l'impression
de passer énormément de temps à écrire des rapports de séances,
rapports d'UA\dots Ajouté au rapport final, cela donne un travail
de rédaction très important.

\subsection{Travail en groupe} % sujet trollesque :D

% git
L'utilisation d'un gestionnaire de version nous a permis de nous
faciliter énormément le travail en centralisant tout le code et les rapports.
Pour certains rapports, nous avons aussi utilisé un «~pad~», qui permet
d'éditer à plusieurs du texte en temps réel.

Concernant le travail en groupe, il est clair que travailler en groupe de 6
en faisant en sorte que tout le monde travaille efficacement n'est pas évident.

Nous avons donc essayé de nous répartir le travail le plus efficacement possible,
mais ça n'a pas non plus évident: à de nombreuses reprises, il fallait se recentrer
et essayer de se re-répartir les tâches clairement, et ce n'est pas évident.
Il y a donc eu certaines séances très peu productives (ou alors seulement
sur certains points) avec aussi des pertes de motivation lors de certaines UAs\dots

Toutefois, les tâches étaient quand même généralement assignées à des binômes ou trinômes,
et le travail a, dans l'ensemble, été plutôt efficace. % LOOOOOOOL

\subsection{Apprentissages}

% on a appris quoi, toussa ?
