% Un bilan devra être présenté dans ce dernier rapport final sur la façon de
% travailler de l'équipe au cours de toutes les UAs. Une autoévaluation devra y
% être conduite de sorte à faire apparaitre les points positifs et/ou les
% disfonctionnements apparus dans l'équipe.

\subsection{Enseignement}

% wtf les séances de TD
Nous avons été surpris de la façon dont fonctionnent les séances de
TD, et avons eu du mal à nous y faire: lors des séances, nous étions
censés être lâchés sur les différents problèmes et nous renseigner par
nous-mêmes.  Toutefois, le professeur présent se mettait souvent à
présenter des choses au tableau, et nous ne savions plus trop si nous
devions avancer, ou stopper tout travail et l'écouter. De plus, il
était souvent difficile de savoir si le professeur s'adressait à un
seul groupe ou à toute la classe; enfin, il n'était généralement pas
trivial de discerner la structure des informations données.

% chacun a quasiment rien vu
Il a aussi été assez frustrant de se répartir le travail: comme nous
étions dans un groupe de 6 et avions quand même un certain nombre de
choses à faire, il fallait forcément se concentrer sur un seul
problème. Du coup, chacun a simplement abordé quelques points précis
sur chaque UA, et n'a pas vraiment acquis de connaissances poussées
sur tout le reste des sujets abordés.

% rapports, rapports...
Le fait de devoir faire des rapports à chaque séance était aussi très
lourd à gérer, et assez peu pratique: nous avions l'impression de
passer énormément de temps à écrire des rapports. Ajouté au rapport
final, cela donne un travail de rédaction très important.

\subsection{Travail en groupe} % sujet trollesque :D

% git
L'utilisation d'un gestionnaire de version nous a permis de nous
faciliter énormément le travail en centralisant tout le code et les
rapports.  Pour les rapports de fin de séance, nous avons aussi
utilisé un «~pad~», qui permet d'éditer à plusieurs du texte en temps
réel (alternative libre et simpliste à Google Docs).

Concernant le travail en groupe, il est clair que travailler en groupe
de 6 en faisant en sorte que tout le monde travaille efficacement
n'est pas évident.

Nous avons donc essayé de nous répartir le travail le plus
efficacement possible mais ça n'a pas non plus évident: à de
nombreuses reprises, il fallait se recentrer et essayer de se
re-répartir les tâches clairement, et ce n'est pas évident.  Il y a
donc eu certaines séances très peu productives (ou alors seulement sur
certains points), avec aussi des pertes de motivation lors de certaines
UAs.

Toutefois, les tâches étaient quand même généralement assignées à des
binômes ou trinômes, et le travail a, dans l'ensemble, été plutôt
efficace.

\subsection{Apprentissage}
  Une chose reste certaine: nous avons tous appris de nouvelles choses pendant
  cette UE, même si elles ne reflètent qu'une partie de ce qui était demandé.

  Le fait de se répartir le travail nous a aussi permis de travailler sur ce
  qui nous intéressait le plus, ce qui est plutôt motivant, quand on a
  eu un sujet qui nous intéressait. Malheureusement, nos centres
  d'intérêts étant souvent les mêmes, il n'était pas rare d'être déçu
  par son affectation. Notamment, nous ne nous sommes pas bousculés
  pour travailler sur Sisyphe, avec raison: ce logiciel s'est avéré
  horrible à prendre en main, et le groupe qui a travaillé sur les
  scarabées a pu coder un programme de tests équivalent en autant de
  temps qu'il a fallu à l'autre groupe pour maitriser cet ``outil''.
