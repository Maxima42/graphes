\subsection{Problème du sac à dos}
  \subsubsection{Présentation du problème}
    Ce problème paraît simple en apparence: nous avons un ensemble d'objets,
    chaque objet pouvant avoir une masse différente et ayant une certaine
    valeur, et nous voulons remplir un sac à dos de manière à maximiser la
    valeur totale, sans dépasser une certaine masse maximale.

    Résoudre ce genre de problème est utile par exemple en gestion de
    portefeuilles pour trouver le meilleur rapport entre rendement et risque,
    ou en découpe de matériaux, pour minimiser les chutes.

    \paragraph{}
    Ce problème est un problème d'optimisation linéaire, en effet, cela revient
    à résoudre le problème:
    \[ \left\{ \begin{array}{l}
        \max v_i \\
        i \in S \\
        \displaystyle\sum_{j \in S} m_j \leq W
      \end{array} \right.
    \]
    où $S$ est l'ensemble des objets à choisir, $v_i$ la valeur de l'objet $i$,
    $m_i$ sa masse et $W$ la masse maximale autorisée dans le sac.

    \paragraph{}
    Cependant la résolution de ce problème n'est pas simple: déterminer s'il
    est possible de dépasser une valeur minimale sans dépasser le poids maximal
    est un problème NP\nobreakdash-complet.

  \subsubsection{Résolution exacte}
    Nous avons implémenté un algorithme de programmation dynamique, qui permet
    de résoudre le problème du sac à dos. Toutefois, il fonctionne uniquement
    si les poids des objets sont des entiers.

    Sa complexité en temps est en $O(nW)$ et celle en mémoire en $O(W)$, avec
    $n$ le nombre d'objets et $W$ le poids maximum du sac.

    Nous l'avons testé\footnote{En utilisant le générateur de problèmes trouvé
    à l'adresse suivante: \url{http://www.diku.dk/~pisinger/codes.html}.} sur
    plusieurs instances du problème, et l'algorithme est rapide.

    %todo: expliquer l'algo

  \subsubsection{Résolution approchée}
    Nous avons aussi implémenté l'algorithme glouton: celui-ci consiste
    simplement à choisir les «~meilleurs~» objets jusqu'à que la masse maximale
    soit dépassée. Le critère déterminant quels sont les meilleurs objets peut
    être la masse faible, le prix élevé, ou le rapport prix/masse élevé.

    Cet algorithme est beaucoup plus rapide que le précédent (il a une
    complexité en temps de $O(n \log n)$ (pour le tri des objets)) et ne
    nécessite en mémoire que la liste des objets, mais ce n'est qu'un algorithme
    approché. Les résultats obtenus sont cependant très satisfaisant, en effet
    en considérant le ratio \nobreak prix/masse, on obtient des résultats très
    proches de l'optimum (quelques pourcents d'erreur relative en moyenne, mais
    aucune garantie n'est fournie).

    De plus il peut être utilisé quand les masses ne sont pas entières.

