\documentclass{scrartcl} \usepackage[utf8]{inputenc}
\usepackage[frenchb]{babel} \usepackage{amssymb} \usepackage{lmodern}
\usepackage[T1]{fontenc} \usepackage{hyperref} \usepackage{verbatim}
\usepackage{listings} \usepackage{graphicx}

\usepackage{color} \definecolor{deepblue}{rgb}{0,0,0.5}
\definecolor{deepred}{rgb}{0.6,0,0}
\definecolor{deepgreen}{rgb}{0,0.5,0}

\newcommand{\brokencell}[2][c]{\begin{tabular}[#1]{@{}c@{}}#2\end{tabular}}

\lstset{frame=single, breaklines=true, breakatwhitespace=true,
basicstyle=\scriptsize, showstringspaces=false, escapeinside={(*}{*)},
keywordstyle=\color{deepblue}, stringstyle=\color{deepred},
commentstyle=\color{deepgreen}, literate= {é}{{\'e}}1{É}{{\'E}}1
{è}{{\`e}}1{È}{{\`E}}1 {ê}{{\^e}}1{Ê}{{\^E}}1 {à}{{\`a}}1{À}{{\`A}}1
{ù}{{\`u}}1{Ù}{{\`U}}1 {û}{{\^u}}1{Û}{{\^U}}1 {ô}{{\^o}}1{Ô}{{\^O}}1
{ó}{{\'o}}1{Ó}{{\'O}}1 {ç}{{\c c}}1{Ç}{{\c C}}1 {œ}{{\oe}}1{Œ}{{\OE}}1
}

\begin{document} \title{Rapport de l'UA4} \author{Maxence Ahlouche
\and Maxime Arthaud \and Korantin Auguste \and Martin Carton \and
Thomas Forgione \and Thomas Wagner} \date{11 novembre
2013} \maketitle \tableofcontents \newpage

\section{Présentation de l'équipe} Cette équipe a été menée par Thomas
Wagner, assisté de son Responsable Qualité Korantin Auguste. Les
autres membres de l'équipe sont Martin Carton, Maxence Ahlouche,
Maxime Arthaud, et Thomas Forgione.

  Tous les membres de l'équipe ont été présents à chacune des séances
lors de cette UA.

\section{Colonie de scarabées}

  Le problème consiste, à partir d'un graphe dont les nœuds
représentent des positions et les arêtes contiennent la probabilité
pour le scarabée de passer d'une position à l'autre, à calculer les
probabilités de présence de chaque scarabée au bout de $N$ itérations,
selon sa position de départ.

  Pour cela, on représente le graphe sous la forme de matrice
d'adjacence, ou chaque élément de la matrice représente la probabilité
de passer d'un nœud à un autre.

  Si on met cette matrice à la puissance N, elle contiendra les
probabilités de passer d'un nœud à un autre en exactement $N$
itérations.

  En multipliant cette matrice par un vecteur contenant uniquement des
zéros, sauf au nœud de départ (on mettra un 1), on peut obtenir la
probabilité pour le scarabée de se trouver à chaque point, au bout
d'exactement $N$ tours.  Si cette probabilité est nulle, il est
impossible qu'il s'y trouve.

  Pour savoir la probabilité que les deux scarabées se rencontrent en
un point au bout de N itérations, il suffit de multiplier les
probabilités de présence de chaque scarabée en ce point. Pour savoir
leur probabilité de se rencontrer en n'importe quel point, on peut
tout simplement sommer les probabilité de rencontre sur chaque point.

  De plus, si on calcule $\lim_{N \to \infty} A^N$ (une telle limite
n'existera que si la chaîne est irréductible, récurrente positive et
apériodique, et ça ne sera pas toujours le cas !), le produit du
vecteur avec la position de départ par cette matrice nous donnera les
probabilités pour le scarabée d'être en chaque position une fois qu'il
aura suffisamment voyagé et que sa position de départ n'aura plus
d'importance.

  \subsection{Chaînes de Markov}

  D'après Wikipedia, le problème des scarabées peut aisément être
modélisé par des chaînes de Markov. En effet, nos coléoptères se
promènent en temps discret dans leur graphe de promenade ; de plus,
l'état d'un scarabée (autrement dit, sa position dans sa promenade) ne
dépend pas du passé, mais uniquement de l'état présent.

  La matrice de transition de notre promenade représente les
probabilités de passer d'un état aux autres. Par conséquent, la
matrice de transition à la puissance $n$ représente la probabilité de
passer d'un état à un autre par un chemin de longueur $n$.

  On constate qu'on se rapproche beaucoup de la matrice décrite
précédemment, de manière si confuse: la matrice d'adjacence du graphe
de promenade et la matrice de transition de notre chaîne de Markov
sont équivalentes!

\subsection{Temps moyen entre rencontre}

\subsubsection{À partir d'un certain point}

Considérons que les deux scarabées sont sur le même point.  On peut
donc calculer la probablité qu'après N tours, ils finissent encore sur
la même case (en sommant le carré des probabilitées que chacun se
retrouve sur une case).

$U_k$ est la probabilité que les deux scarabées se retrouvent sur la
même case après $k$ tours.  $U_k = pA^k(pA^k)^\intercal$

$V_k$ est la probabilité que les deux scarabées se retrouvent sur la
même case après $k$ tours, sans s'être déjà rencontrés avant.

$V_k = (1 - U_1) \times (1 - U_2) \times ... \times (1 - U_{k - 1})
\times U_k$

$V_k$ peut donc se réécrire par récurrence :

$V_1 = U_1$ $V_{k + 1} = V_k \times \frac{1 - U_k}{U_k}U_{k + 1}$

Pour avoir le temps moyen au bout duquel ils vont se rencontrer en
partant de ce même point, il suffit de prendre l'espérance de cette
suite $V_k$.

\end{document}
