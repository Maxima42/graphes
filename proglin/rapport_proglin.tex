\documentclass{scrartcl}
\usepackage[utf8]{inputenc}
\usepackage[frenchb]{babel}
\usepackage{amssymb}
\usepackage{lmodern}
\usepackage[T1]{fontenc}
\usepackage{hyperref}
\usepackage{verbatim}
\usepackage{listings}
\usepackage{graphicx}


\usepackage{color}
\definecolor{deepblue}{rgb}{0,0,0.5}
\definecolor{deepred}{rgb}{0.6,0,0}
\definecolor{deepgreen}{rgb}{0,0.5,0}

\newcommand{\brokencell}[2][c]{\begin{tabular}[#1]{@{}c@{}}#2\end{tabular}}

\lstset{frame=single, breaklines=true,
          breakatwhitespace=true, basicstyle=\scriptsize,
          showstringspaces=false, escapeinside={(*}{*)},
          keywordstyle=\color{deepblue},
          stringstyle=\color{deepred},
          commentstyle=\color{deepgreen},
          literate=
                   {é}{{\'e}}1{É}{{\'E}}1
                   {è}{{\`e}}1{È}{{\`E}}1
                   {ê}{{\^e}}1{Ê}{{\^E}}1
                   {à}{{\`a}}1{À}{{\`A}}1
                   {ù}{{\`u}}1{Ù}{{\`U}}1
                   {û}{{\^u}}1{Û}{{\^U}}1
                   {ô}{{\^o}}1{Ô}{{\^O}}1
                   {ó}{{\'o}}1{Ó}{{\'O}}1
                   {ç}{{\c c}}1{Ç}{{\c C}}1
                   {œ}{{\oe}}1{Œ}{{\OE}}1
        }

\begin{document}
\title{Rapport du projet de programmation linéaire}
\author{Maxence Ahlouche \and Maxime Arthaud \and Korantin Auguste
          \and Martin Carton \and Thomas Forgione \and Thomas Wagner}
\date{21 octobre 2013}
\maketitle
\tableofcontents
\newpage

\section{Présentation de l'équipe}
  Cette équipe a été menée par Maxence Ahlouche, assisté de son Responsable
  Qualité Thomas Wagner. Les autres membres de l'équipe sont Martin Carton,
  Thomas Forgione, Maxime Arthaud, et Korantin Auguste.

  \begin{table}[h]
    \centering
    \begin{tabular}{|c||c|c|c||c|c|c|}
      \hline
      & TD1 & TD2 & TD3 & TP1 & TP2 & TP3 \\
      \hline\hline
      Maxence Ahlouche (CPC) & & & & Abs. & & \\
      \hline
      Maxime Arthaud & & & & & Abs. & \\
      \hline
      Korantin Auguste & Retard & Retard & Retard & Retard & Retard & Retard \\
      \hline
      Carton Martin &&&&&&\\
      \hline
      Thomas Forgione &&&&&&\\
      \hline
      Thomas Wagner (RQ) &&&&&&\\
      \hline
    \end{tabular}
  \end{table}

  Il est à noter qu'en comptant tout ses retards Korantin a loupé au moins 1h45
  de TD/TP :D
  
\section{Problème du sac à dos}
  \subsection{Résolution exacte}
    Nous avons implémenté un algorithme de programmation dynamique, qui permet
    de résoudre le problème du sac à dos.  Toutefois, il fonctionne uniquement
    si les poids des objets sont des entiers.

    Sa complexité en temps est en $O(nW)$ et celle en mémoire en $O(W)$, avec
    $n$ le nombre d'objets et $W$ le poids maximum du sac.

    Nous l'avons testé\footnote{En utilisant le générateur de problèmes trouvé
    à l'adresse suivante: \url{http://www.diku.dk/~pisinger/codes.html}.} sur
    plusieurs instances du problème (jusqu'à X objets et un poids maximal de
    X), et l'algorithme est rapide.
    %todo: faire les tests :D

  \subsection{Résolution approchée}
    Nous avons aussi implémenté l'algorithme glouton: celui-ci consiste à
    choisir les \og meilleurs \fg{} objets jusqu'à que la masse maximale soit
    dépassée.  Le critère déterminant quels sont les meilleurs objets peut
    être la masse faible, le prix élevé, ou le rapport prix/masse élevé.

    Cet algorithme est beaucoup plus rapide que le précédent, mais n'est qu'un
    algorithme approché. La table~\ref{table:greedy} montre quelques-uns des
    résultats obtenus.

    \begin{table}[h]
      \makebox[\textwidth]{%
      \centering
      \begin{tabular}{| c | c | c | c | c | c |}
      \hline
        \brokencell{Paramètres du générateur/\\masse maximale autorisée}
      & \brokencell{Résultat\\optimum}
      & \brokencell{Prix le\\plus élevé}
      & \brokencell{Masse la\\plus faible}
      & \brokencell{Meilleur ratio\\prix/masse}\\
      \hline
      50 25 1 1 1000/20& $85$ & $49/42.4\%$ & $67/21.2\%$ & $81/4.7\%$ \\
      500 25 1 1 1000/500& $2016$ & $1125/44.2\%$ & $1725/14.4\%$ & $1983/1.6\%$ \\
      5000 25 1 1 1000/500& $5540$ & $1175/79\%$ & $4577/17.4\%$ & $5540/0\%$ \\
      50000 25 1 1 1000/500& $11195$ & $1175/90\%$ & $6684/40.3\%$ & $11195/0\%$ \\
      50000 1000 1 1 1000/500& $118260$ & $5959/95\%$ & $101857/13.9\%$ & $118147/0.1\%$ \\
      50000 5000 1 1 1000/100& $100847$ & $14931/85.2\%$ & $93532/7.3\%$ & $100282/0.6\%$ \\
      \hline  
      \end{tabular}
      }
      \caption{Résultats de l'algorithme glouton}
      \label{table:greedy}
    \end{table}

    On remarque qu'en général, la résolution approchée en considérant le ratio
    \nobreak prix/masse donne de très bon résultats, voire le meilleur
    résultat. Le résultat que fournit cet algorithme est le moins bon quand la
    masse maximale autorisée est faible comparée à l'amplitude des valeurs que
    peuvent prendre le prix et la masse des objets.
    
    Toutefois cet algorithme est beaucoup plus rapide, et a une complexité en
    temps de $O(n \log n)$ (pour le tri des objets), et ne nécessite en mémoire
    que la liste des objets.

\section{Problème d'optimisation linéaire} % todo: titre?
  Le but est maximiser une fonction linéaire sous certaines contraintes
  (inéquations linéaires).

  \subsection{Existence de solutions et autre blablas mathématiques}
    \subsubsection{Formalisation du problème}
      Considérons le problème suivant :
      $$ (P) \quad \max_{x\in C \subset \mathbb{R}^n} f(x)$$
      Nous nous placerons dans le cas où $f$ est linéaire, $x \geqslant 0$, 
      et où $C$ est décrit par des contraintes d'inégalités linéaires, 
      c'est-à-dire qu'il existe une matrice $A$ et un vecteur $b$ tels 
      que $Ax\leqslant b$.

    \subsubsection{Existence de solutions}
      Pour un tel problème, trois possibilités s'offrent à nous :
      \begin{itemize}
        \item les contraintes sont incompatibles
        \item la fonction est non majorée sur $C$
        \item le problème admet un maximum sur $C$
      \end{itemize}
      Nous savons de plus que $C$ est un polyèdre convexe. Un théorème garantit
      alors que ce problèma à une solution, alors il a une solution en un de
      ses sommets. Nous allons donc chercher les solutions parmi les sommets de
      $C$.

  \subsection{Algorithme du simplexe}
    Le principe de cet algorithme est de considérer un des sommets du polyèdre,
    puis de se déplacer en suivant les ar\^etes de ce polyèdre en augmentant à
    chaque itération le gain. L'algorihtme se terminera lorsque nous nous 
    trouverons sur un sommet, dont tous les sommets adjacents présentent un gain
    plus faible. La convexité du polyèdre nous garantit que le résultat est 
    optimal.

    L'algorithme du simplexe a une complexité dans le pire des cas
    exponentielle, mais en pratique, cet algorithme est efficace.
    
    % probleme de parenthesage dans ce paragraphe : 3 ouvrantes et 2 fermantes
    Cet algorithme ne permet pas de maximiser une fonction pour des variables
    entières (par exemple pour connaitre le nombre de produits (donc nombre
    entier) à produire pour maximiser un gain (bien qu'on pourrait en pratique
    l'utiliser en considérant que la solution optimale entière est suffisamment
    proche de la solution optimale réelle).

    \subsubsection{Forme standard et tableau canonique}
      Pour résoudre le problème, la première étape est le mettre sous forme
      standard. Pour cela on ajoute à chaque inéquation $j$ de la forme
      $\sum a_{j,i}x_i \leq 0$ une variable dite d'écart pour la transformer en
      égalité: $\sum a_{j,i}x_i + s_j = 0$ où $s_j \geq 0$.
      
      Les inéquations de la forme $\sum a_ix_i \geq 0$ sont d'abord multipliées
      par $-1$ avant cette étape.
      %todo: et les égalités on en fait quoi ? J'ai essayé en les transformant
      %en deux inégalités <= et >= mais ça marche pas.

      On construit ensuite un tableau dit canonique représentant le problème
      comme suit:
      \begin{itemize}
        \item la première ligne de la matrice est
          $[m_0, m_1, \cdots, m_n, 0, \cdots, 0]$ où les $(m_i)$ sont les
          coefficients du problème $\min \sum m_ix_i$ et auquel on ajoute
          autant de $0$ qu'on a ajouté de variables d'écart.
        \item les autres lignes de la matrice sont
          $[a_{j,0}, \cdots, a_{j,n}, 0, \cdots, 0, 1,0, \cdots, 0]$ où les $1$
          sont placés de manière à former une matrice identité (ils
          correspondent aux variables d'écart ajoutées).
      \end{itemize}

    \subsubsection{Algorithme}
      \begin{lstlisting}
Entrée : matrice (un tableau canonique)
Sortie : le résultat optimum
         les quantités de chaque produit à produire
         les quantités de chaque ressources restantes

base = indices des variables de base de la matrice

tant qu'il reste des nombres strictement positifs sur la première ligne:
    à_ajouter = indice de la colonne dont le premier élément est maximal
    à_retirer = indice de la ligne (>1) telle que :
                matrice[à_retirer, dernière colonne]/matrice[à_retirer, à_ajouter]
                est minimum

    remplacer le (à_retirer)ième élément de base par à_ajouter

    diviser la ligne à_retirer de matrice par matrice[à_retirer, à_ajouter]
    et soustraire aux autres lignes y le vecteur :
        matrice[y,a_ajouter] * matrice[a_retirer,] / matrice[a_retirer, a_ajouter]

pour chaque variable d'origine n (n dans [0, nombre de variables hors base[):
  à_produire[base[n]] = matrice[n+1, dernière colonne]

pour chaque variable d'écart n (n dans [nombre de variables hors base, nombre de varibles total[):
  restes[base[n]] = matrice[n+1, dernière colonne]

retourner (-matrice[0, dernière colonne], à_produire, restes)
      \end{lstlisting}

    \subsubsection{Tests}
    %todo
      
    % todo:
    %       contraintes négatives
    %       cas non borné
    \subsubsection{Dégénérescence}
      Un problème du simplexe est dit dégénéré si plus de deux contraintes vont
      devoir être nulles en un sommet. Graphiquement (en 2D), cela veut dire
      qu'au moins 3 droites vont se rencontrer en un sommet.
      %todo : graphiquement ? Pas clair: on a jamais parlé de graphique avant

      Ceci va empêcher l'algorithme du simplexe de progresser entre deux
      itérations: il va simplement changer de base. Le problème étant que sur
      des cas particuliers, il pourra changer de base sans progresser, puis
      boucler à l'infini en faisant un cycle sur des bases qui n'améliorent pas
      la solution.

      Pour éviter cela, on peut utiliser des règles d'anti-cyclage, dont la
      règle de Bland, qui consiste à choisir judicieusement les variables qu'on
      fera entrer et sortir de la base, dans le cas où il y aurait plusieurs
      possibilités autant intéressantes les unes que les autres. Elle consiste
      simplement à se baser sur l'indice des variables. %todo: to be continued?

\section{Annexe}
  \lstlistoflistings
  \lstinputlisting[language=python, caption=Codes relatifs au problème du sac à dos]{knapsack.py}
  \lstinputlisting[language=python, caption=Tests du sac à dos]{knapsack_test.py}
  \lstinputlisting[language=python, caption=Codes relatifs au simplexe]{simplexe.py}


\end{document}
