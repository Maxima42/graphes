\documentclass{scrartcl}
\usepackage[utf8]{inputenc}
\usepackage[frenchb]{babel}
\usepackage{lmodern}
\usepackage[T1]{fontenc}
\usepackage{hyperref}
\usepackage{verbatim}
\usepackage{listings}
\usepackage{graphicx}


\usepackage{color}
\definecolor{deepblue}{rgb}{0,0,0.5}
\definecolor{deepred}{rgb}{0.6,0,0}
\definecolor{deepgreen}{rgb}{0,0.5,0}

\newcommand{\brokencell}[2][c]{\begin{tabular}[#1]{@{}c@{}}#2\end{tabular}}

\lstset{frame=single, breaklines=true,
          breakatwhitespace=true, basicstyle=\scriptsize,
          showstringspaces=false, escapeinside={(*}{*)},
          keywordstyle=\color{deepblue},
          stringstyle=\color{deepred},
          commentstyle=\color{deepgreen},
          literate=
                   {é}{{\'e}}1{É}{{\'E}}1
                   {è}{{\`e}}1{È}{{\`E}}1
                   {ê}{{\^e}}1{Ê}{{\^E}}1
                   {à}{{\`a}}1{À}{{\`A}}1
                   {ù}{{\`u}}1{Ù}{{\`U}}1
                   {û}{{\^u}}1{Û}{{\^U}}1
                   {ô}{{\^o}}1{Ô}{{\^O}}1
                   {ó}{{\'o}}1{Ó}{{\'O}}1
                   {ç}{{\c c}}1{Ç}{{\c C}}1
                   {œ}{{\oe}}1{Œ}{{\OE}}1
        }

\begin{document}
\title{Rapport du projet de programmation linéaire}
\author{Maxence Ahlouche \and Maxime Arthaud \and Korantin Auguste
          \and Martin Carton \and Thomas Forgione \and Thomas Wagner}
\date{21 octobre 2013}
\maketitle
\tableofcontents
\newpage

\section{Présentation de l'équipe}
  Cette équipe a été menée par Maxence Ahlouche, assisté de son Responsable
  Qualité Thomas Wagner. Les autres membres de l'équipe sont Martin Carton,
  Thomas Forgione, Maxime Arthaud, et Korantin Auguste.

  \begin{table}[h]
    \centering
    \begin{tabular}{|c||c|c|c||c|c|c|}
      \hline
      & TD1 & TD2 & TD3 & TP1 & TP2 & TP3 \\
      \hline\hline
      Maxence Ahlouche (CPC) & & & & Abs. & & \\
      \hline
      Maxime Arthaud & & & & & Abs. & \\
      \hline
      Korantin Auguste & Retard & Retard & Retard & Retard & Retard & Retard \\
      \hline
      Carton Martin &&&&&&\\
      \hline
      Thomas Forgione &&&&&&\\
      \hline
      Thomas Wagner (RQ) &&&&&&\\
      \hline
    \end{tabular}
  \end{table}

  Il est à noter qu'en comptant tout ses retards Korantin a loupé au moins 1h45
  de TD/TP :D
  
\section{Problème du sac à dos}
  \subsection{Résolution exacte}
    Nous avons implémenté un algorithme de programmation dynamique, qui permet
    de résoudre le problème du sac à dos.  Toutefois, il fonctionne uniquement
    si les poids des objets sont des entiers.

    Sa complexité en temps est en $O(nW)$ et celle en mémoire en $O(W)$, avec
    $n$ le nombre d'objets et $W$ le poids maximum du sac.

    Nous l'avons testé\footnote{En utilisant le générateur de problèmes trouvé
    à l'adresse suivante: \url{http://www.diku.dk/~pisinger/codes.html}.} sur
    plusieurs instances du problème (jusqu'à X objets et un poids maximal de
    X), et l'algorithme s'exécute toujours en moins d'une seconde.
    %todo: faire les tests :D

  \subsection{Résolution approchée}
    Nous avons aussi implémenté l'algorithme glouton: celui-ci consiste à
    choisir les \og meilleurs \fg{} objets jusqu'à que la masse maximale soit
    dépassée.  Le critère déterminant quels sont les meilleurs objets peuvent
    être la masse faible, le prix élevé, ou le rapport prix/masse élevé.

    Cet algorithme est beaucoup plus rapide que le précédent, mais n'est qu'un
    algorithme approché. La table~\ref{table:greedy} montre quelques-uns des
    résultats obtenus.

    \begin{table}[h]
      \makebox[\textwidth]{%
      \centering
      \begin{tabular}{| c | c | c | c | c | c |}
      \hline
        \brokencell{Paramètres du générateur/\\masse maximale autorisée}
      & \brokencell{Résultat\\optimum}
      & \brokencell{Prix le\\plus élevé}
      & \brokencell{Masse la\\plus faible}
      & \brokencell{Meilleur ratio\\prix/masse}\\
      \hline
      50 25 1 1 1000/20& $85$ & $49/42.4\%$ & $67/21.2\%$ & $81/4.7\%$ \\
      500 25 1 1 1000/500& $2016$ & $1125/44.2\%$ & $1725/14.4\%$ & $1983/1.6\%$ \\
      5000 25 1 1 1000/500& $5540$ & $1175/79\%$ & $4577/17.4\%$ & $5540/0\%$ \\
      50000 25 1 1 1000/500& $11195$ & $1175/90\%$ & $6684/40.3\%$ & $11195/0\%$ \\
      50000 1000 1 1 1000/500& $118260$ & $5959/95\%$ & $101857/13.9\%$ & $118147/0.1\%$ \\
      50000 5000 1 1 1000/100& $100847$ & $14931/85.2\%$ & $93532/7.3\%$ & $100282/0.6\%$ \\
      \hline  
      \end{tabular}
      }
      \caption{Résultats de l'algorithme glouton}
      \label{table:greedy}
    \end{table}
    On remarque qu'en général, la résolution approchée en considérant le ratio
    \nobreak prix/masse donne de très bon résultats, voire le meilleur
    résultat. Le résultat que fournit cet algorithme est le moins bon quand la
    masse maximale autorisée est faible comparée à l'amplitude des valeurs que
    peuvent prendre le prix et la masse des objets.

\section{Problème d'optimisation} % todo: titre?
  Le but est maximiser une fonction linéaire sous certaines contraintes
  (inéquations linéaires).

  \subsection{Simplexe}
    L'algorithme du simplexe a une complexité dans le pire des cas
    exponentielles, mais en pratique, cet algorithme est efficace.

    expliquer principes algo

    % todo:
    %       contraintes négatives
    %       cas non borné
    %       cas non entier

\subsubsection{Dégénérescence}

Un problème du simplexe est dit dégénéré si, plus de deux contraintes vont devoir êtres nulles en un sommet.
Graphiquement (en 2D), cela veut dire que 3 (ou plus) droites vont se rencontrer en un sommet.

Ceci va faire que l'algorithme du simplexe ne va pas progresser entre deux itérations, mais va simplement
changer de base. Le problème étant que sur des cas particuliers, il pourra changer de base sans progresser,
puis boucler à l'infini en faisant un cycle sur des bases qui n'améliorent pas la solution.

Pour éviter cela, on peut utiliser des règles d'anti-cyclage, dont la règle de Bland, qui
consiste à choisir judicieusement les variables qu'on fera entrer et sortir de la base,
dans le cas ou il y aurait plusieurs possibilitées aussi intéressantes les unes que les autres.
Elle consiste simplement à se baser sur l'indice des variables.

\section{Annexe}
  \lstlistoflistings
  \lstinputlisting[language=python, caption=Codes relatifs au problème du sac à dos]{knapsack.py}
  \lstinputlisting[language=python, caption=Codes relatifs au simplexe]{simplexe.py}


\end{document}
