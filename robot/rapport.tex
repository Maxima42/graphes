\documentclass{scrartcl}
\usepackage[utf8]{inputenc}
\usepackage[frenchb]{babel}
\usepackage{hyperref}

\author{Maxence Ahlouche \and Maxime Arthaud \and Korantin Auguste
  \and Martin Carton \and Thomas Forgione \and Thomas Wagner}

\title{Rapport du projet d'ingénierie robotique}
\date{9 décembre 2013}

\begin{document}
\maketitle
\tableofcontents
\newpage

\section{Méthodes de programmation du robot}
\subsection{Programmation graphique}

Au début du projet, nous avons commencé à programmer le robot comme
indiqué dans le sujet, i.e.\ en utilisant le logiciel de programmation
graphique disponible sur Windows. Seulement, ce logiciel, bien
qu'accessible à tout le monde, nous a à tous semblé très peu pratique
à utiliser, si bien que nous n'avons pas été très productifs lors des
premières séances.

En effet, dès que nous souhaitions faire autre
chose que des instructions de base (par exemple, faire tourner le
moteur à une vitesse dépendant de la valeur d'un capteur), nous
passions beaucoup trop de temps à essayer différentes manières
d'implémenter ces instructions sans que notre schéma devienne trop
lourd.

De plus, nous avons rencontré un problème que nous n'avons pas réussi
à résoudre: lorsque nos voulions faire avancer notre robot, il
avançait de manière saccadée. Bien que nous ayons quelques hypothèses
sur l'origine de ce problème, nous n'avons pas réussi à le résoudre de
manière simple via le logiciel fourni.

À cause de ces désagréments, nous avons décidé d'essayer
d'autres méthodes pour programmer le robot.

\subsection{Librairie Python}

Un des membres de notre équipe a déniché sur Internet une librairie en
Python, qui permet de contrôler un robot branché en USB. Cette
librairie (qui s'appelle \texttt{python-nxt}) nous a permis de régler
notre problème d'avancement saccadé.

Elle nous a également permis de programmer notre robot de manière très
simple et très propre. Un programme que nous avions mis plusieurs
heures à implémenter via le logiciel graphique a été réalisé en moins
d'une demi-heure via cette librairie.

Elle présente toutefois un inconvénient majeur: elle ne permet pas (ou alors,
nous n'avons pas trouvé comment faire) de télécharger le programme sur le
robot; par conséquent, le robot devait rester branché à l'ordinateur lors de
l'exécution du programme, et nous devions le suivre avec l'ordinateur. Ce qui
n'est évidemment pas pratique.

\subsection{Langage NXC}

Finalement, nous avons décidé de programmer le robot en utilisant le
langage NXC (\emph{Not Exactly C}), développé spécifiquement pour
les robots NXT.

Bien qu'il ne soit pas facile de trouver de la documentation à jour pour ce
langage, nous avons finalement réussi à l'utiliser proprement.

Le compilateur que nous avons trouvé nous permet également de
télécharger le programme sur le robot, donc nous avons résolu le
problème posé par la librairie en Python, ainsi que ceux posés par le
logiciel de programmation graphique.

Ayant découvert ce langage tardivement dans les séances de TP, nous n'avons
réalisé qu'un seul programme l'utilisant: un programme qui longe les murs.

\section{Longeage de murs}

Comme dit précédemment, nous avons réalisé ce programme en utilisant
NXC.

Nous avons fait le choix, par manque de capteurs, de ne longer que les murs à
gauche du robot.

\subsection{Longeage d'un mur}

Afin de longer un mur, nous avons posé un capteur à ultrasons sur le
côté gauche de notre robot. Lorsque nous détectons que nous sommes
trop éloignés du mur, nous tournons légèrement à gauche.

Cette méthode donnant des résultats peu probants si la direction
initiale du robot n'est pas exactement parallèle au mur, nous avons
décidé d'ajouter un autre capteur à ultrasons à l'avant du robot. Ainsi,
nous tournons désormais à gauche, tout en avançant légèrement, dès
qu'on détecte que le mur est trop loin; et si on se retrouve face à
notre mur, alors on tourne vers la droite jusqu'à ne plus l'être.

Ainsi, nous pouvons suivre un mur droit de manière assez régulière.

\subsection{Passage de l'extérieur d'un angle}

Le passage de l'extérieur d'un angle (i.e.\ quand un mur tourne à
gauche) se fait naturellement à partir du code pour longer d'un mur:
il suffit de tourner à gauche (tout en avançant, afin d'éviter les
collisions avec les murs) jusqu'à ce qu'on soit à nouveau à côté d'un
mur. Naturellement, il a fallu adapter les coefficients de tournage,
afin d'éviter que le robot ne prenne un angle trop large, et se
retrouve finalement face au mur qu'il est censé longer.

\subsection{Passage de l'intérieur d'un angle}

Pour le passage de l'intérieur d'un angle (i.e.\ quand un mur tourne à
gauche), le code découle aussi du premier algorithme. En effet, quand
on se retrouve face à un mur, il suffit de tourner à droite jusqu'à ne
plus détecter de mur face à soi, puis d'avancer.

\subsection{Lissage du mouvement du robot}

Afin d'éviter d'avoir un mouvement erratique lorsqu'on n'est pas
exactement parallèle à la paroi que l'on doit longer (mouvements de
``rebondissement'' sur le mur, par exemple), nous avons décidé
d'améliorer notre algorithme via une astuce très simple: le changement de
direction dépend de la distance du mur. Ainsi, si nous nous éloignons
légèrement du mur (à cause d'une trajectoire non parallèle au mur, par
exemple), alors nous tournerons très légèrement à gauche, sans
ralentir. Ainsi, nous nous retrouverons presque parallèle au mur.

Dans le cas d'un virage à gauche du mur, le capteur détecte un mur à
l'infini; dans ce cas là, nous tournons autant que possible, jusqu'à
retrouver un mur. Dès que le capteur détecte un mur, on repasse dans
le cas précédent; par conséquent, notre robot ne se retrouve quasiment
jamais face au mur qu'il devait longer. De plus, ceci présente
l'avantage que notre robot est désormais capable de faire demi-tour
autour d'un mur qui s'arrête d'un coup, sans virages (comme dans le
cas des planches fournies par notre professeur pour tester le robot).

Dans le cas d'un virage à droite du mur, notre algorithme ne change
pas.

Ainsi, nous avons un robot capable de longer les murs de manière très
fiable, peu importe l'angle des murs.

\end{document}
