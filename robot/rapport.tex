\documentclass{scrartcl}
\usepackage[utf8]{inputenc}
\usepackage[frenchb]{babel}
\usepackage{hyperref}
\usepackage{listings}

\usepackage{color}
\definecolor{blue}{rgb}{0,0,0.4}
\definecolor{red}{rgb}{0.6,0,0}
\definecolor{green}{rgb}{0,0.5,0}
\definecolor{yellow}{rgb}{0.69,0.69,0}

\lstset{frame=single, breaklines=true,
          breakatwhitespace=true, basicstyle=\scriptsize,
          showstringspaces=false, escapeinside={(*}{*)},
          keywordstyle=\color{yellow},
          stringstyle=\color{red},
          commentstyle=\color{green},
          identifierstyle=\color{blue},
          literate=
                   {é}{{\'e}}1{É}{{\'E}}1
                   {è}{{\`e}}1{È}{{\`E}}1
                   {ê}{{\^e}}1{Ê}{{\^E}}1
                   {à}{{\`a}}1{À}{{\`A}}1
                   {ù}{{\`u}}1{Ù}{{\`U}}1
                   {û}{{\^u}}1{Û}{{\^U}}1
                   {ô}{{\^o}}1{Ô}{{\^O}}1
                   {ó}{{\'o}}1{Ó}{{\'O}}1
                   {ç}{{\c c}}1{Ç}{{\c C}}1
                   {œ}{{\oe}}1{Œ}{{\OE}}1
        }

\author{Maxence Ahlouche \and Maxime Arthaud \and Korantin Auguste
  \and Martin Carton \and Thomas Forgione \and Thomas Wagner}

\title{Rapport du projet d'ingénierie robotique}
\date{9 décembre 2013}

\begin{document}
\maketitle
\tableofcontents
\newpage

\section{Présentation de l'équipe}
Cette équipe a été menée par Maxime Arthaud, assisté de son Responsable
Qualité Thomas Forgione. Les autres membres de l'équipe sont Martin Carton,
Maxence Ahlouche, Korantin Auguste et Thomas Wagner.

\section{Méthodes de programmation du robot}
\subsection{Programmation graphique}

Au début du projet, nous avons commencé à programmer le robot comme
indiqué dans le sujet, c'est à dire en utilisant le logiciel de programmation
graphique disponible sur Windows. Seulement, ce logiciel, bien
qu'accessible à tout le monde, nous a semblé très peu pratique % pas à tout le monde, mais à ceux qui paye une licence pour un mauvais OS
à utiliser, si bien que nous n'avons pas été très productif lors des
premières séances.

En effet, dès que nous souhaitions faire autre
chose que des instructions de base (par exemple, faire tourner le
moteur à une vitesse dépendant de la valeur d'un capteur), nous
passions beaucoup trop de temps à essayer de comprendre comment fonctionnait
l'interface.

De plus, nous avons rencontré un problème que nous n'avons pas réussi
à résoudre: notre robot, avançait de manière saccadée. Bien que nous ayons
quelques hypothèses sur l'origine de ce problème (bug ou documentation non à
jour), nous n'avons pas réussi à le résoudre avec le logiciel fourni.

À cause de ces désagréments, nous avons décidé d'essayer
d'autres méthodes pour programmer le robot.

\subsection{Librairie Python}

Un des membres de notre équipe a trouvé sur Internet une librairie en
Python, qui permet de contrôler un robot branché en USB. Cette
librairie (qui s'appelle \texttt{nxt-python}) nous a permis de régler
notre problème d'avancement saccadé.

Elle nous a également permis de programmer notre robot de manière très
simple, et de gagner beaucoup de temps par rapport au logiciel de
programmation graphique.

Elle présente toutefois un inconvénient majeur: elle ne permet pas de télécharger
le programme sur le robot (le python étant interprété sur l'ordinateur);
par conséquent, le robot devait rester branché à l'ordinateur lors de
l'exécution du programme, et nous devions le suivre avec l'ordinateur. Ce qui
n'est évidemment pas pratique.

\subsection{Langage NXC}

Finalement, nous avons décidé de programmer le robot en utilisant le
langage NXC (\emph{Not Exactly C}), développé spécifiquement pour
les robots NXT.

Le compilateur que nous avons trouvé nous permet également de
télécharger le programme sur le robot, donc nous avons résolu le
problème posé par la librairie en Python, ainsi que ceux posés par le
logiciel de programmation graphique.

\section{Suivie de murs}

Nous avons réalisé un programme qui permet au robot de longer un mur, grâce
à NXC.

Nous avons fait le choix, par manque de capteurs, de ne longer que les murs à
gauche du robot.

\subsection{Suivi d'un mur}

Afin de longer un mur, nous avons posé un capteur à ultrasons sur le
côté gauche de notre robot. Lorsque nous détectons que nous sommes
trop éloignés du mur, nous tournons légèrement à gauche.

Cette méthode donnant des résultats peu probants si la direction
initiale du robot n'est pas exactement parallèle au mur, nous avons
décidé d'ajouter un autre capteur à ultrasons à l'avant du robot. Ainsi,
notre robot ne fonce plus dans les murs, et tourne à droite quand le
capteur de devant détecte quelque chose.

Ainsi, nous pouvons suivre un mur droit de manière assez régulière.

\subsection{Lissage du mouvement du robot}

Pour avoir un déplacement plus fluide de notre robot, avec moins
d'à-coups, nous avons utilisé la technique de \emph{régulateur PID},
avec seulement l'action proportionnelle (P).

Le PID est une technique de correction d'erreur utilisé couramment en
robotique. Il s'agit d'un algorithme calculant le signal à émettre
en fonction des mesures. Connaissant la mesure et la valeur voulue,
on en déduit l'erreur. La correction de cette erreur se fait par 3
actions :

\begin{itemize}
    \item une action proportionnelle \textit{P} : on multiplie l'erreur par une constante
    \item une action intégrale \textit{I} : on intègre l'erreur, et on la divise par une constante
    \item une action \textit{D} : on dérive l'erreur, et on la multiplie par une constante
\end{itemize}

Finalement, on somme les 3 actions pour obtenir la variation à appliquer à notre
commande.

Concrètement, dans notre cas, il nous à suffit de faire la différence entre
la distance mesurée par le capteur de gauche et la distance voulue, puis
de multiplier ceci par une constante. On ajoute cette valeur à la puissance initiale
des moteurs pour avoir un déplacement proportionnelle à la distance au mur.
Avec cette méthode, nous avons un robot se déplaçant de manière très fluide.

Pour de vrai robot, les constantes multiplicatives dans le PID demandent des réglages
fins, car ils influent sur l'efficacité de l'algorithme. Dans notre cas, il nous
a suffit de tester avec moins d'une dizaine de valeur pour trouver quelque chose
de satisfaisant.

\subsection{Recherche d'un mur}

Dans le cas où le robot est placé loin d'un mur, il va se mettre à tourner sur lui même.
Nous avons donc eu l'idée de faire en sorte qu'il fasse plutôt une spirale pour
rechercher le mur le plus proche.

\section{Labyrinthe}
  Pour résoudre le problème du labyrinthe, nous avons simplement utilisé le
  suivi de mur gauche. En effet cela permet de sortir d'un labyrinthe sans
  ilots, même si ce n'est pas forcement efficace.

  Nous avons aussi réfléchi à deux algorithmes plus efficaces: Dijkstra (que
  nous avions déjà implémenté dans l'UA1) et A*.  Nous n'avons pas implémenté
  ces algorithmes sur le robot, cas nous ne savions pas comment avec un robot
  Lego nous aurions pu réaliser le parcourt du labyrinthe en enregistrant une
  carte de celui-ci. Or ces algorithmes nécessitent que le labyrinthe soit
  connu.

\section{Annexes}
\lstlistoflistings
\lstinputlisting[language=c, caption=Robot suiveur de murs]{suivre_mur.nxc}

\end{document}
