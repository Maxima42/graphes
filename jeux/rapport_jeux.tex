\documentclass{scrartcl}
\usepackage[utf8]{inputenc}
\usepackage[frenchb]{babel}
\usepackage{amssymb}
\usepackage{lmodern}
\usepackage[T1]{fontenc}
\usepackage{hyperref}
\usepackage{verbatim}
\usepackage{listings}
\usepackage{graphicx}


\usepackage{color}
\definecolor{deepblue}{rgb}{0,0,0.5}
\definecolor{deepred}{rgb}{0.6,0,0}
\definecolor{deepgreen}{rgb}{0,0.5,0}

\newcommand{\brokencell}[2][c]{\begin{tabular}[#1]{@{}c@{}}#2\end{tabular}}

\lstset{frame=single, breaklines=true,
          breakatwhitespace=true, basicstyle=\scriptsize,
          showstringspaces=false, escapeinside={(*}{*)},
          keywordstyle=\color{deepblue},
          stringstyle=\color{deepred},
          commentstyle=\color{deepgreen},
          literate=
                   {é}{{\'e}}1{É}{{\'E}}1
                   {è}{{\`e}}1{È}{{\`E}}1
                   {ê}{{\^e}}1{Ê}{{\^E}}1
                   {à}{{\`a}}1{À}{{\`A}}1
                   {ù}{{\`u}}1{Ù}{{\`U}}1
                   {û}{{\^u}}1{Û}{{\^U}}1
                   {ô}{{\^o}}1{Ô}{{\^O}}1
                   {ó}{{\'o}}1{Ó}{{\'O}}1
                   {ç}{{\c c}}1{Ç}{{\c C}}1
                   {œ}{{\oe}}1{Œ}{{\OE}}1
        }

\begin{document}
\title{Rapport du projet de programmation linéaire}
\author{Maxence Ahlouche \and Maxime Arthaud \and Korantin Auguste
          \and Martin Carton \and Thomas Forgione \and Thomas Wagner}
\date{11 novembre 2013}
\maketitle
\tableofcontents
\lstlistoflistings
\newpage

\section{Présentation de l'équipe}
  Cette équipe a été menée par Maxence Ahlouche, assisté de son Responsable
  Qualité Thomas Wagner. Les autres membres de l'équipe sont Martin Carton,
  Thomas Forgione, Maxime Arthaud, et Korantin Auguste.

  Todo si nécessaire, torm sinon:
  \begin{table}[h]
    \centering
    \begin{tabular}{|c||c|c|c||c|c|c|}
      \hline
      & TD1 & TD2 & TD3 & TP1 & TP2 & TP3 \\
      \hline\hline
      Maxence Ahlouche (CPC) & & & & & & \\
      \hline
      Maxime Arthaud & & & & & & \\
      \hline
      Korantin Auguste &&&&&& \\
      \hline
      Carton Martin &&&&&&\\
      \hline
      Thomas Forgione &&&&&&\\
      \hline
      Thomas Wagner (RQ) &&&&&&\\
      \hline
    \end{tabular}
  \end{table}

\section{Shifumi}
\section{Morpion}
\section{Compétition/Duopole}
  blabla

  \subsection{Stratégies}
    \subsubsection{Coopératif}
    Todo: pourquoi 0.75? Avec $\frac{7-\sqrt{13}}{4}$ on a les mêmes résultats.

    \subsubsection{Non-coopératif}
      Cette stratégie consiste à maximiser ses gains par rapport à ce que joue
      (ou a joué au tour précédent) l'autre joueur.
      Todo: blabla

    \subsubsection{Stackelberg}
      Todo: Pourquoi 2/3? Pourquoi 1.1*2/3 c'est mieux.

      Une variante de cette stratégie (voir listing~\ref{lst:stackmean})
      consiste à utiliser la production moyenne de l'«~adversaire~» plutôt que
      seulement la dernière valeur. Elle permet d'obtenir des résultats
      légèrement meilleurs.

    \subsubsection{Cournot}
      Todo: à comprendre + à faire en matlab

    \subsubsection{Pénalise}
      Le principe de cette stratégie (voir listing~\ref{lst:penalise}) est
      d'être coopératif tant que l'adversaire l'est, et de devenir plus
      agressif quand il ne l'est plus: à chaque fois que l'«~adversaire~» n'est
      pas coopératif, on joue comme le ferait la stratégie Stackelberg.

      Une variante de cette stratégie (voir listing~\ref{lst:penaliseviolent})
      consiste à le pénaliser de plus en plus: la première fois on le pénalise
      une fois, puis deux, puis trois, etc.

      Ces deux stratégies sont efficaces à la fois quand l'autre joueur est
      coopératif (on est alors coopératif) et contre un joueur non-coopératif
      (on devient alors agressif).

\section{Annexes}
  \lstinputlisting[label=lst:stackmean, language=matlab, caption=Statégie stackelbergmean1.m]{duopole/stackelberg_mean1.m}
  \lstinputlisting[label=lst:penalise, language=matlab, caption=Statégie penalise1.m]{duopole/penalise1.m}
  \lstinputlisting[label=lst:penaliseviolent, language=matlab, caption=Statégie penalise\_violent1.m]{duopole/penalise_violent1.m}
  
\end{document}
